\newpage
\begin{enumerate}
	\item \textbf{Introduction}
	\begin{enumerate}[label*=\arabic*.]
		\item \textbf{Problem}
		\item \textbf{Concepts}
			\item \textbf{Contributions}
			\item \textbf{Structure of the Report}
		\end{enumerate}
		\item \textbf{Background}
		\begin{enumerate}[label*=\arabic*.]
			\item \textbf{NoSQL}
			\item \textbf{RDF}
			\item \textbf{Statistics}
			\begin{enumerate}[label*=\arabic*.]
				\item \textit{Population and Sample}
				\item \textit{Distributions}
				\item \textit{Correlation}
				\item \textit{Regression}
			\end{enumerate}
			\item \textbf{Graph Theory}
			\begin{enumerate}[label*=\arabic*.]
				\item \textit{Metrics}
				\item \textit{Random-Graph}
				\item \textit{Scale-Free}
				\item \textit{Watts-Strogatz}
				\item \textit{Hierarchical}
			\end{enumerate}
		\end{enumerate}
	
	
	\item \textbf{Related Works}
	\begin{enumerate}[label*=\arabic*.]
		\item \textbf{Graph Analysis}
		\begin{enumerate}[label*=\arabic*.]
			\item \textit{Studies of Barabási and Albert}
			\item \textit{Network Robustness and Metric Correlations}
			\item \textit{Chinese Network Analysis}
			\item \textit{Conclusion}
		\end{enumerate}
		\item \textbf{NoSQL Benchmarks} or RDF / SPARQL Benchmarks
		\begin{enumerate}[label*=\arabic*.]
			\item \textit{YCSB}: maybe its not necessary
			\item \textit{Berlin}
			\item \textit{DBpedia}
			\item \textit{SP2}
			\item \textit{Conclusion}
		\end{enumerate}
	\end{enumerate}
	
	
	
	\item \textbf{Train Benchmark Framework}
	\begin{enumerate}[label*=\arabic*.]
		\item \textbf{Main Concepts}
		\item \textbf{Workflow}
		\item \textbf{The Domain}
		\item \textbf{Supported Formats}
		\item \textbf{Performance Comparison and Uniformity}
		\item \textbf{Tools}
		\item \textbf{Framework's Architecture}
	\end{enumerate}

	
	\item \textbf{Contributions}
	\begin{enumerate}[label*=\arabic*.]
		\item \textbf{Disadvantages of the Train Benchmark Domain} (necessary?)
		\item \textbf{A Real-life Model}
		\begin{enumerate}[label*=\arabic*.]
			\item \textit{Overview of a real-life Data: Train Schedules}
			\item \textit{Mapping to a Model}
			\item \textit{Model Analysis}
		\end{enumerate}
		\item \textbf{Extending the Framework with Analysis}
		\begin{enumerate}[label*=\arabic*.]
			\item \textit{Metrics Calculation}
			\item \textit{The New Architecture of the Framework}
			\item \textit{Workflow}
			\item \textit{Regression Analysis}
		\end{enumerate}
		\item \textbf{Model Generation}
		\begin{enumerate}[label*=\arabic*.]
			\item \textit{Concepts}
			\item \textit{Overall of the Generation Steps}
			\item \textit{Step 1: Topologies} (maybe omit "Step 1" from the title, it's strange)
			\begin{enumerate}[label*=\arabic*.]
				\item Random Graph
				\item Watts-Strogatz Model
				\item Scale-free Networks of Barabási-Albert
				\item Hierarchical Network
			\end{enumerate}
			\item \textit{Step 2: Schedule Connections}
			\item \textit{Possible Model Configurations}
			\item \textit{Uniform Model Generation}
		\end{enumerate}
		\item \textbf{The Workloads for Graph Queries Analysis}
		\begin{enumerate}[label*=\arabic*.]
			\item \textit{Main Goal}
			\item \textit{Sample Choosing}
			\begin{enumerate}[label*=\arabic*.]
				\item A Sample Based on Topologies
				\item A Sample Based on Metrics
			\end{enumerate}
			
			\item \textit{Model Configuration}
			\item \textit{Evaluated Queries}
			\item \textit{Test the Performance Estimation}
			\item \textit{Investigated Tools}
		\end{enumerate}
	\end{enumerate}
	
	
	\item \textbf{Evaluation}
	\begin{enumerate}[label*=\arabic*.]
		\item \textbf{Benchmarking Environment}
		\item \textbf{How to Read the Charts}
		\item \textbf{Multiple Regression Analysis}
		\item \textbf{MARS Analysis}
		\item \textbf{Performance Estimation}
		\item \textbf{Conclusions}
	\end{enumerate}
	\item \textbf{Summary}
	\begin{enumerate}[label*=\arabic*.]
		\item \textbf{Future Work}
	\end{enumerate}
	\item \textbf{Acknowledges}
\end{enumerate}






\newpage
\begin{enumerate}
	\item \textbf{Introduction}
		\begin{enumerate}[label*=\arabic*.]
			\item \textbf{Problem}\\
			In model oriented systems (not MDSD), performance is a key role.
			Databases, NoSQL systems -> increasing data -> importance of optimization.\\
			
			performance depends on workloads -> a bit different model + query implicate a high deviation in performance\\
			Real-time response. Important to estimate response time to optimize.
			
			There is no benchmark framework for NoSQL / RDF that investigates workload and performance relationship precisely, they only concentrate on performance. Only one static model can be found in these frameworks.
			

			\item \textbf{Concepts}\\
			A framework that systematically assess and analyze performance.
			Why is it good: utilizations in query optimizers.
	
			+ \textbf{\textbf{Fig}}: show the typical workflow of our goal: \\
			The challenge is to find regression models. \\
			Analyze models by descriptive metrics -> find connections.
			
			\item \textbf{Contributions}\\
			A brief overview of the work -> graph topologies with different distributions based on a real model-> metrics -> analyze performance, estimation\\
			Regression analysis.
			
			Arbitrary size, topology, density, and cardinality on the generated models.
			
			Test the framework's usability, test the performance estimation: arbitrary workload -> try to match a regression model. Can it be used in optimizations?
			\item \textbf{Structure of the Paper}
		\end{enumerate}
	\item \textbf{Background}
		\begin{enumerate}[label*=\arabic*.]
			\item \textbf{NoSQL} - briefly
			\item \textbf{RDF} - short, maybe 1 \textbf{\textbf{Fig}} of an example
			\item \textbf{Statistics}
				\begin{enumerate}[label*=\arabic*.]
					\item \textit{Population and Sample} - or sample only
					\item \textit{Distributions}\\ implicate the definition of random variable.
							Uniform; (normal?); poisson; pareto - power-law
					\item \textit{Correlation}\\implicate covariance definition. \\
						Pearson definition + formula\\
						+1 \textbf{\textbf{Fig}} of different samples and their correlations.
					\item \textit{Regression}\\linear, multiple briefly, give equation
					+ 1 \textbf{\textbf{Fig}}
				\end{enumerate}
			\item \textbf{Graph Theory}
				\begin{enumerate}[label*=\arabic*.]
					\item \textit{Metrics}\\Degree Distribution;clustering coefficient+\textbf{Fig}, average path length, betwenness centrality
					\item \textit{Random-Graph}\\generation algorithm, \textbf{Fig}, typical metric values, distribution
					\item \textit{Scale-Free}\\generation algorithm, \textbf{Fig}, typical metric values, distribution
					\item \textit{Watts-Strogatz}\\generation algorithm, \textbf{Fig}, typical metric values, distribution
					\item \textit{Hierarchical}\\generation algorithm, \textbf{Fig}, typical metric values, distribution
				\end{enumerate}
		\end{enumerate}
		
		
	\item \textbf{Related Works}\\Introduce the field of graph analysis and sparql benchmarks. Goal is to merge them in our search.
		\begin{enumerate}[label*=\arabic*.]
			\item \textbf{Graph Analysis}
				\begin{enumerate}[label*=\arabic*.]
					\item \textit{Studies of Barabási and Albert}\\
						Scale-free can be found in real-life networks.
					\item \textit{Network Robustness and Metric Correlations}\\
						Failure of one metric to estimate robustness. They tried to find metric - connectivity, robustness connection. They use the same models.
					\item \textit{Chinese Network Analysis}\\
						Well, it's not so interesting. They search correlations in the model, but show small-word and scale-free characteristics in their model.
					\item \textit{Conclusion}\\We can see the models and metrics that worth to use, and appear in real networks\\
						The failure of one metric by searching performance/robustness correlations -> need to involve more metrics to find relationships -> use regression.
				\end{enumerate}
			\item \textbf{NoSQL Benchmarks} or RDF / SPARQL Benchmarks
				\begin{enumerate}[label*=\arabic*.]
					\item \textit{YCSB}: use different distributions in the workloads
					\item \textit{Berlin}: real life use cases, some performance metrics
					\item \textit{DBpedia}: same performance metrics like in berlin + real-model
					\item \textit{SP2}: systematic queries -> try to cover every important problem in sparql
					\item \textit{Conclusion}\\summarize the benchmarks + Table\\
					Some of them measure metrics, but they only cover the performance like evaluated queries per second\\
					They use one static model in different sizes. (YCSB tells nothing about its model.)
					Draw conclusions for our framework -> we concentrate on the model's structure, and generate various topologies instead of using one static model. And we explore model - performance relationships.
				\end{enumerate}
		\end{enumerate}
		
		
		
	\item \textbf{Train Benchmark Framework}
		\begin{enumerate}[label*=\arabic*.]
			\item \textbf{Main Concepts} \\ maybe a \textbf{Fig} of a model validation
			\item \textbf{Workflow}\\+ \textbf{Fig} about the phases and the roles of models, queries, and constraints -> how the queries can be attached to model validations
			\item \textbf{The Domain} \\\textbf{Fig} of the metamodel
			\item \textbf{Supported Formats}
			\item \textbf{Performance Comparison and Uniformity}
			\item \textbf{Tools} - table, briefly
			\item \textbf{Framework's Architecture} \\+\textbf{Fig}, roles of the components
		\end{enumerate}
		
		
	\item \textbf{Contributions}
		\begin{enumerate}[label*=\arabic*.]
			\item \textbf{Disadvantages of the Train Benchmark Domain} (necessary?)\\ 
			\textbf{Fig} of the cardinalities of the railway domain -> important to recall the railway metamodel again
			\item \textbf{A Real-life Model}
				\begin{enumerate}[label*=\arabic*.]
					\item \textit{Overview of a real-life Data: Train Schedules}\\ records, main attributes, cardinalities
					\item \textit{Mapping to a Model}\\ +\textbf{Fig} metamodel or ER diagram of the domain (EMF metamodel is already done).
					This section is about the abstraction and not the model generation.
					\item \textit{Model Analysis} \\
					Original data characteristics\\
					+ \textbf{Fig} showing the degree distributions of schedules.
				\end{enumerate}
			\item \textbf{Extending the Framework with Analysis}
			\begin{enumerate}[label*=\arabic*.]
				\item \textit{Metrics Calculation}: model and query metrics
				\item \textit{The New Architecture of the Framework} \\
						summon the same figure from section 4.7, but extend it: Analyzer, QueryBuilder, Generator Class. More emphasis on analysis components, than generation -> that will be described below.
				\item \textit{Workflow}\\
						+ \textbf{Fig} about the phases.\\
						Reuse model metrics, dynamic queries.
				\item \textit{Regression Analysis}\\ 
					How we analyze the measurements? multiple regression by R + MARS
			\end{enumerate}
			\item \textbf{Model Generation}\\ 
				\begin{enumerate}[label*=\arabic*.]
					\item \textit{Concepts}\\
						Idea: dynamic internal structures. Emphasize model structure. Study model metric - performance connection. Allude the other benchmark frameworks again.
					\item \textit{Overall of the Generation Steps}\\
						7 steps + \textbf{Fig}
					\item \textit{Step 1: Topologies} (maybe omit "Step 1" from the title, it's strange)
						Do not define these topologies, they are already done in 2.4. But tell specific details, extensions about their algorithms. \\
						Important: why are we using these? because of their metrics
						\begin{enumerate}[label*=\arabic*.]
							\item Random Graph: G(n,p) model
							\item Watts-Strogatz Model: extended: $N$ value is not a constant
							\item Scale-free Networks of Barabási-Albert: extended: $m$ is not a constant neither\\
								Maybe refer to the heterogeneous model.
							\item Hierarchical Network\\
								challenge -> finish the algorithm to get $N$ nodes
								(a \textbf{Fig} would be great about a small model)
						\end{enumerate}
					\item \textit{Step 2: Schedule Connections} \\
						Generate power-law distr. from uniform + Formula
						Use BFS algorithm.
					\item \textit{Possible Model Configurations}\\
						User can change the model size, topology among stations, cardinality, and density.
						Maybe + Table about the model sizes and triples.
					\item \textit{Uniform Model Generation}\\
						We can generate models with the same average degrees, in the same size -> so edges are equal.
						Add the hierarchical network's formula for estimating edges, since it is the key to achieve uniformity.
						+ \textbf{Fig} about this metric.
				\end{enumerate}
			\item \textbf{The Workloads for Graph Queries Analysis}
				\begin{enumerate}[label*=\arabic*.]
					\item \textit{Main Goal}\\
						Investigate graph queries performance -> how the model structures and their metrics affect the performance. And also study, how useful is the benchmark, how can its regressions be used for optimizations.
					\item \textit{Sample Choosing}
							\begin{enumerate}[label*=\arabic*.]
								\item A Sample Based on Topologies\\ 5 model from every topology -> bad metric deviations\\ +\textbf{Fig}
								\item A Sample Based on Metrics\\
									use 1-1 Hier, 1 scale, and 3 WS model -> better metric deviations for our goal \\+\textbf{Fig}
							\end{enumerate}
					
					\item \textit{Model Configuration}\\
							change stations proportions, from 1.8 to 80% 
					\item \textit{Evaluated Queries}\\
							Transitive Closure query\\
								+ sparql definition
							Navigations query\\
							(Attributes query? maybe)
					\item \textit{Test the Performance Estimation}\\
						test with arbitrary models
					\item \textit{Investigated Tools} - which one supports transitive closure?
				\end{enumerate}
		\end{enumerate}
		
		
	\item \textbf{Evaluation}
			\begin{enumerate}[label*=\arabic*.]
				\item \textbf{Benchmarking Environment}
				\item \textbf{How to Read the Charts}
				\item \textbf{Multiple Regression Analysis}
				\item \textbf{MARS Analysis}
				\item \textbf{Performance Estimation}
				\item \textbf{Conclusions}
			\end{enumerate}
	\item \textbf{Summary} - a 5 pages list of my achievements
		\begin{enumerate}[label*=\arabic*.]
			\item \textbf{Future Work}
		\end{enumerate}
	\item \textbf{Acknowledgements}
\end{enumerate}


