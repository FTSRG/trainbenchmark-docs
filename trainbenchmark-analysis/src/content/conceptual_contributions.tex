\chapter{Design}

\section{Overview of the Approach}

In this section, we introduce the main notions about our work to analyze model and performance relationships.

Our concept includes the following systems illustrated in Figure \ref{fig:frameworks}. We rely on two existing frameworks in our work: the Train Benchmark and MONDO-SAM frameworks. Furthermore, we the elaborate \framework (MONDO-Metrics-Based Analysis of Performance) and the Homogeneous Graphs Benchmark.
%todo mondo sam cite
\begin{figure}[!ht]
	\centering
	\includegraphics[width=130mm, keepaspectratio]{figures/frameworks.pdf}
	\caption{An overview of the frameworks in our approach.}
	\label{fig:frameworks}
\end{figure}

\paragraph{MONDO-SAM}
The MONDO-SAM framework was created under the project MONDO (Scalable Modeling and Model Management on the Cloud)~\cite{mondo} with the motivation of providing a common benchmark framework in Model Driven Engineering (MDE) for benchmark developers~\cite{mondo-sam}. \sam can be considered as an abstract layer that proposes an evaluation engine to execute arbitrary workflows independently on the current workload. Furthermore, \sam also provides tools for serializing and visualizing the benchmark results.

\paragraph{Train Benchmark}
The Train Benchmark framework is based on the evaluation engine provided by \sam, and it proposes a benchmark for measuring continuous model validations and transformations. The Train Benchmark is presented in detail in Section \ref{sec:train}.

\paragraph{\framework}
The goal of the \map framework is to support model-based performance analysis. To achieve this, it extends \sam with additional features, as it provides a framework for analyzing model and performance relationships in the light of an arbitrary workload by characterizing the performance quantitatively with model related metrics. Similarly to \sam, \map is an abstract framework and can be extended by an arbitrary benchmark case.

\paragraph{Homogeneous Graphs Benchmark}

\begin{figure}[!ht]
	\centering
	\includegraphics[width=140mm, keepaspectratio]{figures/approach.pdf}
	\caption{The main concept of the metric-based performance analysis.}
	\label{fig:approach}
\end{figure}

The Homogeneous Graphs Benchmark (HG Benchmark) extends the \map framework with a benchmark case for RDF databases. It generates homogeneous graphs---well-known network topologies---and it investigates the relationships between the model metrics and performance of query evaluations with respect to an arbitrary query and tool. The goal is to generate artificial graphs with various characteristics and obtain a spread in their descriptive metrics, and thus showing a quantitative connection between model metrics and performance. As Figure \ref{fig:frameworks} suggests, in the HG Benchmark we use a part of the components of the Train Benchmark that are adequate for our purpose as well.

Figure \ref{fig:approach} illustrates the main concept of our work. First, the HG Benchmark generates different $K$ graph topologies. Using the graph metric definitions from \map (2.), we calculate the descriptive metrics for every topology in step 3. In the next two steps (4-5.) we define a query and evaluate it on every topology in the system under benchmark. The measurement result is the query evaluation time ($Y$) that represents another variable belonging to the topologies besides their metrics (6.). As it was mentioned before, the \sam framework is responsible for publishing the benchmark results (7.). Finally, in \map we analyze the results by creating regression models to estimate the influence of metrics to the performance.

%\subsubsection{Challenges}

%Three challenges can be found in our research. At first, we have to provide a solution in the HG benchmark to generate models with different %characteristics that impact the performance of a particular query and cause a deviation in evaluation time. 
%Second, we have to find model metrics that are able to characterize the models individually. Furthermore, these attributes must be suitable to %create quantitative relationships among them and the evaluation times.

\section{Models and Metrics}
\subsection{Real-Life Networks}

%todo copy heavy tail to background
In the field of graph theory, the internal structures of real-life networks are comprehensively investigated. The main approach in the analysis of these networks is to explore the degree distributions and study the specific metrics---typically the clustering coefficients, average degrees, average shortest paths---that are suited to characterize the graphs appropriately. Based on the degree distributions and metrics, one can draw a conclusion how a particular real-life network shows a similar characteristic to the well-known topologies such as the random graph, scale-free model, small-world model of Watts-Strogatz or the hierarchical network.\\
For example, the network of World-Wide-Web is studied in~\cite{www1} and~\cite{www2} as well, and the authors observe that the degree distribution of the \textit{www} follows a power-law distribution with a heavy tail, which indicates the presence of web pages with significantly higher degrees than the average degree. Since the probability of occurrences of these web pages is considerably low, the connectivity of the world-wide-web can be represented by the scale-free model of Barabási and Albert.

More examples can be found in the study of Barabási and Albert~\cite{statistical_mechanics}, as they review the advances of different publications and investigate the characteristics of different real-life networks. Empirical results prove that the \textit{movie actor collaboration network}, \textit{cellular networks}, \textit{phone call} and \textit{citation} networks also follow power-law distributions. Many of the studied networks can be considered as scale-free models, however, a part of these graphs---for example the network of movie actors---also show small-world properties and a high clustering coefficient in their connectivity similarly to the Watts-Strogatz or hierarchical topology.

As far as the Watts-Strogatz model---and the random graph---are concerned, their specific degree distribution---Poisson---rarely appears in the real-life networks, as it is emphasized in~\cite{random_study}. However, the small-world property of the WS models frequently appears in the real-life networks. In practice, none of the artificial topologies can be identified perfectly to real-life models, however, the representative metrics of these artificial networks can be observed in real-life networks as well. 

We assume that those metrics that show high deviations among the different topologies are suited to characterize and identify them individually, furthermore, they may able to characterize an arbitrary graph as well. If these metrics are capable to characterize entirely different networks, then we assume that these metrics are the key to characterize the performance of query evaluations.

\subsection{Network Topologies and Representative Metrics} \label{sec:topology_metric}

Barabási and Albert inspect the natures of the well-known graph topologies in~\cite{statistical_mechanics}, such as the random graph, scale-free and the Watts-Strogatz model. As a main result, they observe that there are significant differences among the topologies regarding specific graph metrics. Based on their study and the research of hierarchical graphs~\cite{hierarchical}, the following metric deviations are assumed between the four topologies, illustrated in Table \ref{tab:topology_metrics}. The random graph is considered as a reference point, and every value is compared to its metrics by assuming that the networks are in the same size.
\begin{table}[ht]
	\footnotesize
	\centering
	\begin{tabular}{ l c c c c}
		\toprule
		Metric & Random & Hierarchical & Scale-free & Watts-Strogatz \\ 
		\midrule 
		\textbf{Max Degree} & $\bullet$ & $\bullet$$\bullet$$\bullet$ & $\bullet$$\bullet$$\bullet$ & $\bullet$ \\ \hline
		\textbf{Clustering Coefficient} & $\bullet$ & $\bullet$$\bullet$$\bullet$ & $\bullet$ & $\bullet$$\bullet $\\ \hline
		\textbf{Avg Shortest Path Length} & $\bullet$ & $\bullet$$\bullet$ & $\bullet$ & $\bullet$$\bullet$ \\ \hline
		\bottomrule
	\end{tabular}
	\caption{Graph topologies and their descriptive metrics}
	\label{tab:topology_metrics}
\end{table}
%todo describe bullets in detail
As Table \ref{tab:topology_metrics} demonstrates, each topology can be characterized by different metric values, which leads to the assumption that if the diversity between the topologies may cause high variance in the performance of a particular query evaluation, then these metrics are adequate to characterize the model and performance relationships quantitatively.

%todo use lattice ref in background
%todo emphasize lattice, random and p value in background, maybe add a picture as well

However, the values of metrics in Table \ref{tab:topology_metrics} are misleading due to the reason that the metrics of the Watt-Strogatz model highly depend on the initialization of the network, namely, the value of $p$ probability that is used in the generation.

\begin{figure}[!ht]
	\centering
	\includegraphics[width=100mm, keepaspectratio]{figures/ws_metrics.png}
	\caption{Characteristic path length $L(p)$ and clustering coefficient $C(p)$ of Watts-Strogatz model~\cite{ws_metrics}}
	\label{fig:ws}
\end{figure}

By modifying $p$, the Watts-Strogatz model represents a bridge between a lattice and a random graph. As Figure \ref{fig:ws} illustrates, the clustering coefficient $C(p)$ and the average shortest path ($L(p)$) metrics are changed with respect to $p$ scaling. The values are normalized by $L(0)$ and $C(0)$ that represent the clustering coefficient and average shortest path metrics for a lattice graph. As a conclusion, the Watts-Strogatz model shows significant deviations in these two metrics in the light of the $p$ probability value. 

%todo add betweenness reference to background
Besides the models in Table \ref{tab:topology_metrics}, the metrics also require modifications. The problem is that the maximum degree metric alone does not include a comprehensive information about the internal structure of the network, since it does not emphasize the role of a node with maximum degree in the connectivity of the graph. Hence, we use another metric---the \textit{betweenness centrality}---which characterizes adequately the importance of a higher degree, since the higher value of betweenness centrality belongs to a vertex, the more shortest paths include that node, symbolizing that node represents the center in the graph.

After these modifications, the topologies and the related metrics are shown in Table \ref{tab:topology_metrics2}. WS-$p$ indicates the Watts-Strogatz model with a probability value of $p$. The values of the betweenness centrality are determined by our initial assumption considering that the \textit{center} node in a hierarchical network and the \textit{hubs} in a scale-free model may occur more times in the shortest paths due to the fact that they have higher degrees. The main conclusion is that we can achieve a higher deviation among the metric values by using these topologies, and thus, in the following we concentrate on these networks.
\begin{table}[ht]
	\footnotesize
	\centering
	
	\begin{tabular}{ l c c c c c c}
		\toprule
		Metric & Random & Hierarchical & Scale-free & WS-0.1 & WS-0.01 & WS-0.001 \\ 
		\midrule 
		\textbf{Max Degree} & $\bullet$ & $\bullet$$\bullet$$\bullet$ & $\bullet$$\bullet$$\bullet$ & $\bullet$ & $\bullet$ & $\bullet$ \\ \hline
		\textbf{Clustering Coefficient} & $\bullet$ & $\bullet$$\bullet$$\bullet$ & $\bullet$ & $\bullet$$\bullet$ & $\bullet$$\bullet \bullet$ & $\bullet$$\bullet$$\bullet$\\ \hline
		\textbf{Avg Shortest Path Length} & $\bullet$ & $\bullet$$\bullet$ & $\bullet$ & $\bullet $ & $\bullet \bullet$ & $\bullet$$\bullet$$\bullet$\\ \hline
		\textbf{Betweenness Centrality} & $\bullet$ & $\bullet$$\bullet$$\bullet$ & $\bullet$$\bullet$ & $\bullet$ & $\bullet$ & $\bullet$\\ \hline
		\bottomrule
	\end{tabular}
	\caption{Graph topologies and their descriptive metrics with extensions}
	\label{tab:topology_metrics2}
\end{table}

\section{Metric and Performance Comparison}

Showing performance and metric relationship is an essential goal of our approach. The first notion is the search of correlations between the metrics and performance.

A similar problem is studied in~\cite{algebraic1} and~\cite{algebraic2}, where the authors generate well-known topologies and inspect the connectivity and robustness of the networks. In their case, a network is said to be robust if its performance is not sensitive to the changes in topology. In~\cite{algebraic1}, the \emph{algebraic connectivity} (not discussed in detail in this report) metric is studied to search robustness and metric relationships, however, they show that the algebraic connectivity is not trivially correlated to the robustness of the network.

The authors in~\cite{algebraic2} investigate the impact of betweenness centrality, algebraic connectivity and average degree to robustness, and they also draw the conclusion that there is no unique graph metric to satisfy both connectivity and robustness objectives while keeping a reasonable complexity, since each metric captures some attributes of the graph.

Partly based on the advances of these two publications and our initial assumption of the topologies and their metrics (Table \ref{tab:topology_metrics2}), we expect that we cannot find a correlation between one metric and the performance, hence, we suspect that only the ensemble of more metrics is suited to find relationship.

In order to find quantitative connections, we use regression analysis to show how the various metrics impact the performance. 

\subsection{Choosing the Sample}

By using regression analysis on a sample, it is important to regard the sample to be unbiased. In our case, a bias in a sample of graph topologies means a variation in the size of the models. Obviously, one topology in the sample with larger amount of nodes can bias the connection between the models and the performance, therefore, our framework must support the generation of \textit{uniform} models\footnote{From now on, under the concept of uniform models we mean models with approximately equal number of nodes and edges, despite the diversity of their internal structures.} with respect to the amount of nodes and edges, even in the case of different topologies.
