\pagenumbering{roman}
\setcounter{page}{1}

\selectlanguage{magyar}
%\hungarianParagraph

\englishParagraph
%----------------------------------------------------------------------------
% Abstract in Hungarian
%----------------------------------------------------------------------------
\chapter*{Kivonat}\addcontentsline{toc}{chapter}{Kivonat}
A modell központú rendszerek tervezése során kulcsfontosságú a modellen kiértékelt lekérdezések optimális teljesítménye, ennek elérése érdekében pedig a megfelelő végrehajtó eszközök megválasztása. Az elmúlt években különböző NoSQL adatbázis-kezelő rendszerek váltak népszerűvé, amelyek célja a gyors lekérdezés kiértékelés és skálázhatóság biztosítása.

A lekérdezés kiértékelésének teljesítményét nagy mértékben befolyásolja a modell topológiája és a lekérdezés komplexitása. Célunk az, hogy bizonyos, a modellt és a lekérdezést leíró metrika felhasználásával, kapcsolatot találjunk a metrikák és a teljesítmény között, úgy, hogy az adott metrikákat ismerve, a teljesítmény megjósolható legyen.

A metrikák alapján tervezési szintű döntéseket hozhatunk az optimális teljesítményre törekedve, továbbá, ez a tudás lehetőséget teremt a valósidejű lekérdezés optimalizálás területén is arra, hogy a modellt és a lekérdezést jellemző metrikák alapján döntsünk optimalizálást érintő kérdésekben.

Különböző NoSQL adatbázis-kezelő rendszereket vizsgálva, regressziós analízis segítségével olyan metrikák után kutatunk, amelyek képesek az adott rendszer futási teljesítményét jellemezni. A vizsgálandó adathalmazok előállításához, egy valós modellből kiindulva, különböző eloszlású és topológiájú gráfokat generálunk.

További célunk az, hogy eldöntsük, vajon egy tetszőleges struktúrájú adatmodellre képesek vagyunk-e a regressziós analízisünk alapján a teljesítményre becslést adni, illetve a modellhez a megfelelő, készletünkben lévő adatbázis-kezelő rendszert társítani, a legjobb teljesítményre koncentrálva. 

Végezetül, a valós idejű lekérdezések optimalizálásától motiválva, választ keresünk arra, hogy egy tetszőleges adatmodellre, költséghatékony  metrikaszámítással a teljesítmény előre megbecsülhető-e.



\vfill
\selectlanguage{english}
\englishParagraph


%----------------------------------------------------------------------------
% Abstract in English
%----------------------------------------------------------------------------
\chapter*{Abstract}\addcontentsline{toc}{chapter}{Abstract}
Achieving the optimal performance of query evaluations plays an important goal in model-based systems. In the last few years, different NoSQL database systems were introduced to provide a better performance and solve the problem of scalability.

The performance of query evaluations depends on the topology of the model, and the complexity of the particular query. Our primary goal is that --- by defining model and query-related quantitative metrics ---  to find a considerable connection between metrics and the performance, and thus, be able to predict the query evaluation time.

Based on the metrics and their effect to the performance, we are able to make decisions in design to achieve on the optimal performance. Furthermore, this knowledge can be utilized in the area of real-time query optimization engines as well.

We investigate various NoSQL database systems via regression analysis in order to find different model-related metrics that are suited to characterize their performance appropriately. Based on a real model, we generate graphs with various topologies and distributions to find metrics from different aspects.

Furthermore, we explore that whether an arbitrarily structured model's performance is predictable via our regression analysis, and also predict which database system from our scope can be associated to the model in order to achieve an optimal performance.

Finally, being motivated by the real-time optimization engines, we search answers whether by reducing the cost of metric calculations, an arbitrary model's performance is still predictable.


\vfill
\dolgozatnyelve
\defaultParagraph

\newcounter{romanPage}
\setcounter{romanPage}{\value{page}}
\stepcounter{romanPage}