\section{Analysis of results}
\label{sec:results:analysis}

\subsection{Batch validation}

% all characteristics are polynomial, slopes are similar: mostly constant factor differences
% dominated by model de-serialization for simpler queries
% 
% three groups of tools (within 3-4 orders of magnitude)
% slow: MySQL, Neo4j, Neo4jRAM, Drools
% medium: Sesame, Java
% fast: OCL, IncQuery

As it can be seen in the plots of \autoref{sec:results:batchvalidation}, the overall execution time characteristics of all tools are similar. With the exception of the \emph{SemaphoreNeighbor} query, even the (asymptotic) slopes are identical, which means that there is only a constant multiplier difference between the various tools. The \emph{SemaphoreNeighbor} query proved to be significantly more complex than the others (as indicated by metrics evaluation results) as the slopes for some tools (Java, Eclipse OCL and Drools) are larger, indicating an exponential disadvantage compared to the fastest tools (\eg Neo4j, \eiq and Sesame).

Taking a closer look at the dominantly constant-multiplier differences for queries \emph{PosLength}, \emph{SwitchSensor} and \emph{RouteSensor}, the measured tools can be approximately grouped into three categories:
\begin{itemize}
  \item The \emph{fastest tools} are Eclipse OCL and \eiq, which is no surprise since they operate on in-memory structures and use non-trivial optimizations supported by their evaluation engines.
  \item Sesame, Java and Drools qualify as the \emph{medium} of the spectrum, explained by the fact that they use in-memory structures but lack the optimization capabilities of \eiq and Eclipse OCL. This is somewhat surprising for Drools as it is advertised to incorporate optimization capabilities, yet this is not seen in our results.
  \item MySQL and Neo4j correspond to the \emph{slower end}, possibly explained by their use of less optimized disk-oriented data structures and underoptimized query evaluation engines (especially the Cypher engine in the case of Neo4j). The usage of RAM disks can alleviate the speed disadvantage only insignificantly.
\end{itemize}

\paragraph{IncQuery vs. OCL specific observations.}

% In case of batch query evaluation, both OCL implementations use the same
% algorithm, thus their execution time is roughly the same. The roughly negligible
% differences are due to the initialization of the OCL Impact Analyzer.
 
From the \emph{batch query} evaluation of the \textsf{PosLength}, \textsf{SwitchSensor} and \textsf{RouteSensor} queries (\autoref{sec:results:batchvalidation}), it can be seen that \eiq performs similarly to Eclipse OCL. It is slightly faster for small models (2 s and 3 s respectively), but is slower for large models (up to 125 s and 78 s), where this 50\% slowdown (once in the whole scenario) can be attributed to the initial (Rete) cache build.

% Both OCL measurements use Eclipse OCL for the batch phase,
% because the Impact Analyser is not built for this purpose according to the
% authors. However, the initialisation of the OCL Impact Analyser also occurs in
% this phase resulting in (roughly negligible) higher execution time.

However, for the more complex \textsf{SemaphoreNeighbor} query, it is observed that \eiq outperforms Eclipse OCL by a large margin solutions: it is noticeably faster for small modells (2 s and 4 s), and over 435k model elements OCL did not finish with the initial analysis in 12 minutes. This performance gain might be attributed to the more efficient (cached) enumeration of instances, and the possibility of backward navigation (with the help of auxiliary structures) on unidirectional references used by this query.

\subsection{Incremental revalidation}
As designed, this phase illustrates the characteristic difference between traditional, ``batch'' query engines and incremental tools clearly. \emph{Total revalidation time} (shown on the X axis in the plots of \autoref{sec:results:revalidation}) corresponds to the execution time of editing and re-checking operations, \ie it approximates an incremental model transformation sequence for incremental tools and a batch transformation + batch query evaluation sequence for traditional tools.

% characteristic difference between search-based and incremental tools
% 
% fast: IncQuery, Drools. IncQuery much faster for complex queries

The characteristic advantage of incremental tools (\ie \eiq and Drools) is best seen on the Y axes of the plots, as these tools are typically 1-2 orders of magnitude (\ie 10 to 100 times) faster than the rest. Since the plots include editing and result retrieval operations that scale linearly with model size, the overall low-order polynomial characteristics apply to all of the tools. The \textsf{SemaphoreNeighbor} query again highlights \eiq's clear advantage when dealing with very complex queries: it beats all other tools by 3-5 orders of magnitude.


% \paragraph{IncQuery vs. OCL-specific observations}
% 
% increase in time explained by increasing result set which must be read completely
% (best seen for \emph{SemaphoreNeighbor})

% In the \emph{incremental case}, Eclipse OCL evaluates the query on each run
% (100 times) from scratch, its execution time increases linearly with
% model size, resulting slow overall evaluation.
 
% For the \textsf{RouteSensor} query (\autoref{fig:SumInc_RouteSensor}), the Impact
% Analyzer performs the 100 modifications in 350 ms regardless of the model
% size. On the same query, \eiq starts much faster, but its speed reduces
% on the larger models (from 9 to 220 ms). On the other hand, the Impact
% Analyzer is an order of magnitude slower on the \textsf{SemaphoreNeighbor} query
% (\autoref{fig:SumInc_SemaphoreNeighbor}) query: it does not finish in 12 minutes
% for models over 61k model elements, while \eiq handles every model
% regardless of size under 40 ms.
 
% The performance of the Impact Analyzer is most likely affected by the previously
% mentioned unidirectional references. The slowdown of \eiq is probably
% caused by the increased number of matches (from 116 to 8592), as query
% results are always available in the output nodes of Rete networks, and only a
% linear traversal of these stored matches is needed to return them.

\subsection{Highlights of Interest}

\paragraph{Differences between the Inject and Repair scenarios.}
As noted in \autoref{sec:transformatios}, the main difference between the \textsf{Inject} and \textsf{Repair} scenarios is the amount of model manipulation operations, which is significantly larger for the \textsf{Repair} scenario. The query result sets are also larger for the \textsf{Repair} scenario. By comparing corresponding plots, it is observed that the overall evaluation time is affected linearly by this difference (explained by the specification which requires a complete reading through the entire result list), meaning that all tools are capable of handling this efficiently.

\paragraph{Execution time differences can be huge.}
While the overall characteristics of all tools are similar (low order polynomial with a constant component), we have recorded a rather large variation of execution times (0.5--4 orders of magnitude difference). This confirms our expectation that model persistence, (de)serialization, query evaluation and model transformation technologies can have a huge overall impact on MDE performance. 

\paragraph{RAM disks matter less than anticipated.}
Some measurements using 4store and Neo4j were performed with both SSD-based and RAM disk-based storage backends. Overall, as it is seen from total time measurements (\autoref{sec:results:totaltime}), RAM disk-based variants were about twice as fast -- which is less than we originally anticipated.

\paragraph{Hand-coded Java and MySQL are surprisingly fast in some cases.}
We created the hand-coded Java and MySQL implementations as baselines to which more sophisticated tools can be measured. While the advanced tools are the clear winners overall, the Java and MySQL baselines were surprisingly fast in several configurations. We interpret this as a sign that more sophisticated tools, and specifically MDE technologies still have a lot of performance enhancement potential to be unlocked.

\paragraph{JIT HotSpot optimizations kick in late.}
As seen in \autoref{sec:results:seriesedit} and even better in \autoref{sec:results:seriesrevalidation}, the execution time for certain operations executed in cycles can decrease dramatically over time. In the Java Runtime environment, this effect is best explained by Just-in-time (HotSpot) compiling by the Java Virtual Machine. Overall, it can be concluded that at least 30--50 complex operation iterations are advised to be executed in a benchmarking environment to obtain results that are relevant to long running software.

\subsection{Threats to Validity}

\subsubsection{Internal threats}

\paragraph{Mitigating measurement risks.}
We tried to mitigate \emph{internal validity} threats by reducing the number of uncontrolled variables during measurements: a physical machine was used with all unnecessary software components turned off and every execution was isolated into a freshly initialized JVM.

\paragraph{Ensuring functional equivalence and correctness.}
The queries are semantically equivalent in the different query languages and the result sets are the same for every model. Our current measurements only loaded and executed a single query in each run. When loading multiple queries, query interference may change the results greatly. A more detailed evaluation of this issue is planned for the future.

% \paragraph{Memory measurements.}
% It is important to note that heap usage were measured after executing a garbage collection, so these measurements do not contain memory usage of temporary  constructs. This means that maximum heap usage might have been larger. Furthermore, limiting heap space by the maximum usage results in excessive garbage collection and thus an increased runtime. However, in our experience setting the limit to two times the measured values, such issues do not occur.

\paragraph{Code reviews.}
Additionally, to ensure comparable results the created high-quality query implementations were reviewed: the OCL implementation by Ed Willink from the Eclipse OCL project, the usage of Impact Analyzer by Axel Uhl from the Impact Analyzer developer team. The graph patterns were written by the developers of \eiq. SPARQL code was reviewed by a semantic web expert. Drools rules were reviewed by the JBoss Drools developer team.

\subsubsection{External threats}

\paragraph{Metrics and generalizability of results.}
Considering \emph{external validity}, the generalization of the results largely depends on how representative the metamodels, the models and the queries are compared to real use cases. In section \ref{specification-extended} metrics were defined to describe complexity of models and queries, however comparing them to real world ones remains a future work.

\paragraph{Industrial relevance.}
The metamodel and the query specifications were motivated by an industrial case study, and the selected queries feature commonly used validation tasks such as attribute and reference checks or cycle detection. We tried to ensure that the instance models have a similar structure and distribution to other models by parameterizing the generation process based on our experience with other domains. To summarize, we believe that the train domain and the generated instance models represent other domain-specific languages and available instance models well.

% \paragraph{Resource usage.}
% Considering resource-constrained environments, we believe that limiting available memory will alter the results the most, as the memory management overhead will reduce the performance of \eiq.
% In case of the benchmark queries, we measured a $1.5$GB heap size in case of a model with $3.5$M model elements that we believe is manageable in a developer
% machine with $4$--$8$GB of RAM. On the other hand, when handling such large models the existing user interface itself could become a bottleneck. Thus we believe, our measurement results hold also in the integrated development environments.
