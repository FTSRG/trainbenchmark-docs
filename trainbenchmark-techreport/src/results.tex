\chapter{Benchmark Results}


\section{Benchmarking Environment}
\label{sec:environment}

For the implementation details, source codes and raw results, see the benchmark website\footnote{\url{http://incquery.net/publications/benchmarkmetrics}}. In this section we describe the runtime environment, and highlight some design decisions.

The benchmark machine contains two quad-core Intel Xeon L5420 (2.50~GHz) CPU, 32~GBs of RAM and an SAS disk formatted to ext4 for storing the models. In order to alleviate disturbance of a running measurement and minimize noise in the results, a bare metal 64-bit Ubuntu 12.04 LTS OS was installed with unnecessary services (like \texttt{cron}) turned off. OpenJDK JVM version 1.6.0\_24 is used as the Java environment and Eclipse Kepler Modeling 64-bit for Linux to satisfy specific tool dependencies.

The performance measurements of a tool was independent from the others, i.e.\ for every tool only its codebase was loaded, and every measurement of a scenario was started in a different JVM. Before the execution, the OS file cache was cleared, and swapping was disabled to avoid this kind of thrashing. Each test case (including all phases) must be run within a specified time limit (15 minutes), otherwise its process was killed.

Before acquiring memory usage (free heap space) from the JVM, GC calls were triggered five times to sweep unfreed objects from the RAM. For a JVM, 25~GB heap limit was specified, and to compensate 64-bit pointers, OOPS (ordinary object pointers) compression was also turned on with the following options: (\code{-XX:MaxPermSize=256M -XX:+UseCompressedOops -Xmx25G}).

In the benchmark all cases were run 10 times, and the results were dumped into files, which were aggregated using the R statistical framework. The correlation results and performance plots are written into an HTML report.


\section{Tools}
\label{tools}
The measured tools generally work on graph-based models (like EMF~\cite{EMF} or RDF~\cite{RDF}), and provide a graph pattern-like query language. \autoref{tbl:tools} shows the list of current implementations.

\begin{table}[h]
	\centering
	\footnotesize
	\begin{tabular}{  | l | r | l | m{1.4cm} | c | >{\centering}m{1.9cm} | m{2.3cm} | }
	\hline
	\multicolumn{1}{|c|}{\bf Tool} & 
	\multicolumn{1}{c|}{\bf Version} & 
	\multicolumn{1}{c|}{\bf Model DL} & 
	\bf Query language & 
	\multicolumn{1}{c|}{\bf Incremental} & 
	\bf In-memory only & 
	\bf Implementation language \\ \hline 
	Java & 7.0 & EMF & Java & \ding{109} & \ding{108} & C++ \\ \hline
	Eclipse OCL & 3.3.0 & EMF & OCL & \ding{109} & \ding{108} & Java \\ \hline
	%OCL Impact Analyzer & 3.1.0 & EMF & OCL & \ding{108} & \ding{108} & Java \\ \hline
	\incquery{} & 0.7.2 & EMF & IQPL & \ding{108} & \ding{108} & Java \\ \hline
	Drools & 5.4.0 & EMF & DRL & \ding{108} & \ding{108} & Java \\ \hline
	Sesame & 2.7.9 & RDF & SPARQL & \ding{109} & \ding{108} & Java \\ \hline
	4store & 1.1.5 & RDF & SPARQL & \ding{109} & \ding{109} & C \\ \hline
	Neo4j & 1.9.2 & Graph & Cypher & \ding{109} & \ding{109} & Java \\ \hline
	\end{tabular}
	\caption{Tools used in the benchmark.}
	\label{tbl:tools}
\end{table}


\subsection{EMF-based Tools}

\subsubsection{Java}
An imperative \emph{local search-based} approach was implemented in Java, operating on \emph{Eclipse Modeling Framework (EMF)}~\cite{EMF} models. Queries are implemented as Java functions, traversing a model without any search plan optimization, but they cut unnecessary search branches at the earliest possibility.

\subsubsection{Eclipse OCL}

The OCL~\cite{OCL} language is commonly used for querying EMF model instances in validation frameworks. It is a standardized navigation-based query language, applicable over a range of modeling formalisms. Taking advantage of the expressive features and wide-spread adoption of this query language, the project Eclipse OCL~\cite{EclipseOCL} provides a powerful query interface that evaluates such expressions over EMF models.

%\subsection{OCL Impact Analyzer}
%The Eclipse OCL project also supports incremental evaluation by including an Impact Analyzer (IA)~\cite{ocl-ia2} that calculates the constraints to be reevaluated based on a model change. During EMF modifications it looks for possible context objects that could change the match set, and re-evaluation can be executed only for those objects. As it is intended only for incremental use, basic Eclipse OCL is used for calculating the first result set (batch mode).

\subsubsection{\incquery{}}
\incquery{}~\cite{models10} is an Eclipse Modeling project that provides incremental query evaluation using Rete~\cite{rete} nets. Queries can be written in its graph pattern based query language (IncQuery Pattern Language, IQPL~\cite{iqpl}), which is evaluated over EMF models.

\subsubsection{Drools}
Incremental query evaluation is also supported by the \concept{Drools}~\cite{Drools} rule engine developed by Red Hat. It is based on a variant of Rete~\cite{rete} (object-oriented Rete). Queries can be formalized using its own rule description language. Queries can be constructed by naming the ''when'' part of rules and acquiring their matches. While Drools is not an EMF tool per se, the Drools implementation of the \tb{} works on EMF models.

\subsection{RDF-based Tools}

\subsubsection{Sesame}
Sesame gives an API specification for many tools, and also provides its own implementation. The tool evaluates queries over RDF that are formulated as SPARQL~\cite{Sparql} graph patterns.

\subsubsection{4store}
4store~\cite{harris20094store} is an open source, distributed triplestore implemented in C. The main goal of 4store is to provide a high performance storage and query engine for semantic web applications. 

\subsection{Graph-based Tools}

\subsubsection{Neo4j}
As part of the NoSQL movement, database management systems emerged with a focus on graph storage and processing. As of 2014, the most popular graph database is 
Neo4j~\cite{neo4j}. The data model is based on graphs, where any node or edge can be labeled. Cypher can be used to query labeled graphs using its own graph pattern notation. This engine also uses disk for data storing, so a RAM disk is created during the benchmark.

\section{Measurement Results for Performance Comparison}
\label{sec:results}

The measurement results of the benchmark are displayed below. These diagrams show the batch query performance, incremental evaluation time, and memory usage of each tools, for different model sizes. Additionally, the initial and the updated result set size is displayed under the model sizes for the batch and incremental queries, respectively.

%The left column shows charts of the \textsf{RouteSensor} query,
%while the more complex \textsf{SignalNeighbor} is presented in the right column.
%The remaining \textsf{PosLength} and \textsf{SwitchSensor} queries are only presented
%at the benchmark website~\cite{TBwebsite}, as their results are very similar to the
%\textsf{RouteSensor} case.

\subsection{Batch Validation (User Scenario)}

\forallqueries{\benchmarkresult{User_ReadPCheck0_\n}{Batch validation times for the \textsf{\n} query in the \textsf{User} scenario.}}

\subsection{Batch Validation (XForm Scenario)}

\forallqueries{\benchmarkresult{XForm_ReadPCheck0_\n}{Batch validation times for the \textsf{\n} query in the \textsf{XForm} scenario.}}

\subsection{Revalidation (User Scenario)}

\forallqueries{\benchmarkresult{User_SumModifyPSumReCheck_\n}{Revalidation times for the \textsf{\n} query.}}

\subsection{Revalidation (XForm Scenario)}

\forallqueries{\benchmarkresult{XForm_SumModifyPSumReCheck_\n}{Revalidation times for the \textsf{\n} query in the \textsf{XForm} scenario.}}

\subsection{Total Time (User Scenario)}

\forallqueries{\benchmarkresult{User_SumTime_\n}{Total time for the \textsf{\n} query in the \textsf{User} scenario.}}

\subsection{Total Time (XForm Scenario)}

\forallqueries{\benchmarkresult{XForm_SumTime_\n}{Total time for the \textsf{\n} query in the \textsf{XForm} scenario.}}

\subsection{Memory Usage (User Scenario)}

\forallqueres{\benchmarkresult{User_Memory_\n}{Memory consumption of the \textsf{\n} query in the \textsf{User} scenario.}}

\subsection{Memory Usage (XForm Scenario)}

\forallqueries{\benchmarkresult{User_Memory_\n}{Memory consumption of the \textsf{\n} query in the \textsf{XForm} scenario.}}

\subsection{Series of Edit Times (User Scenario)}

\benchmarkresult{series_DetEdit_64_PosLength}{Series of edit times for the \textsf{PosLength} query in the \textsf{User} scenario.}
\benchmarkresult{series_DetEdit_16_RouteSensor}{Series of edit times for the \textsf{RouteSensor} query in the \textsf{User} scenario.}
\benchmarkresult{series_DetEdit_1_SignalNeighbor}{Series of edit times for the \textsf{SignalNeighbor} query in the \textsf{User} scenario.}
\benchmarkresult{series_DetEdit_128_SwitchSensor}{Series of edit times for the \textsf{SwitchSensor} query in the \textsf{User} scenario.}

\subsection{Series of Incremental Revalidation Times (User Scenario)}

\benchmarkresult{series_DetCheck_64_PosLength}{Series of incremental revalidation times for the \textsf{PosLength} query in the \textsf{User} scenario.}
\benchmarkresult{series_DetCheck_16_RouteSensor}{Series of incremental revalidation times for the \textsf{RouteSensor} query in the \textsf{User} scenario.}
\benchmarkresult{series_DetCheck_1_SignalNeighbor}{Series of incremental revalidation times for the \textsf{SignalNeighbor} query in the \textsf{User} scenario.}
\benchmarkresult{series_DetCheck_128_SwitchSensor}{Series of incremental revalidation times for the \textsf{SwitchSensor} query in the \textsf{User} scenario.}

