
% For a basic description of the transformations, see https://trac.inf.mit.bme.hu/Ontology/wiki/TrainBenchmarkTODO#a20120712
% user
%   result set size = ~10 (small constant)
%   #edit = ~10 (small constant)
% modify:
%   - introduce errors (SwitchSensor, RouteSensor) or eliminates errors (PosLength, SignalNeighbor) with the modification of random instance elements
%   - can modify any instance elements, not only elements from the result set
%   - edit phase: 10*(edit, check)
%   - timeout: 10 minutes (total time) (user's coffee break)

\section{\tb{} Queries}

\picSmall{trainbenchmark-metamodel}{The metamodel of the \tb{}}

In the following, we present the queries defined in the \tb{}. %These queries were widely used during the development of \iqd{}, both for functional tests and performance benchmarks of different query engines, including \iqd{} itself. 

We describe the semantics and the goal of each query. We also show the associated graph pattern and relational algebra query. The metamodel of the railroad system is shown in \figref{trainbenchmark-metamodel}.

\subsection{Relational Schemas}

For formulating the queries in relational algebra we define the following relational schemas for representing the vertices (objects) in the graph (instance model).

\begin{itemize}
  \item $ \mathit{Route}\left(\mathit{id}\right) $
  \item $ \mathit{Sensor}\left(\mathit{id}, \mathit{Segment\_length}\right) $
  \item $ \mathit{Signal}\left(\mathit{id}\right) $
  \item $ \mathit{Switch}\left(\mathit{id}\right) $
  \item $ \mathit{SwitchPosition}\left(\mathit{id}\right) $
  \item $ \mathit{TrackElement}\left(\mathit{id}\right) $
\end{itemize}

The edges (relationships) are represented with the following relational schemas:

\begin{itemize}
  \item $ \mathit{Route\_entry}\left(\mathit{Route}, \mathit{Signal}\right) $
  \item $ \mathit{Route\_exit}\left(\mathit{Route}, \mathit{Signal}\right) $
  \item $ \mathit{Route\_switchPosition}\left(\mathit{Route}, \mathit{SwitchPosition}\right) $
  \item $ \mathit{Route\_routeDefinition}\left(\mathit{Route}, \mathit{Sensor}\right) $
  \item $ \mathit{SwitchPosition\_switch}\left(\mathit{SwitchPosition}, \mathit{Switch}\right) $
  \item $ \mathit{TrackElement\_sensor}\left(\mathit{Switch}, \mathit{Sensor}\right) $
  \item $ \mathit{TrackElement\_connectsTo}\left(\mathit{TrackElement}, \mathit{TrackElement}\right) $
\end{itemize}

\subsection{Graph Patterns}

Blue rectangles and arrows mark simple constraints, while red rectangles and arrows represent negative application conditions. The query return with the nodes in hollow blue rectangles. Additional constraints (e.g.\ arithmetic comparisons) are shown in the figure in text.

%%%%%%%%%%%%%%%%%%%%%%%%%%%%%%%%%%%%%%%%%%%%%%%%%%%%%%%%%%%%%%%%%%%%%%%%%%%%%%%
\subsection{PosLength}

\subsubsection{Description} The \textit{PosLength} well-formedness constraint requires that a segment must have positive length. Therefore, the query (\figref{trainbenchmark-poslength}) checks for segments with a length less than or equal to zero. The SPARQL representation of the query is shown in \lstref{poslength-sparql}.

\subsubsection{Goal} The query checks whether an object has an attribute. If it does, the value is checked. Checking attributes is a real world use case, although a very simple one. Note that simple string checking is also measured in the Berlin SPARQL Benchmark \cite{BSBM}, and it concludes that the string comparison algorithm dominates the query time.

\lstset{language=SQL,morekeywords={PREFIX,FILTER,OPTIONAL,BOUND}}

\begin{lstlisting}[caption=The PosLength query in SPARQL, label=lst:poslength-sparql]
PREFIX base: <http://www.semanticweb.org/ontologies/2011/1/TrainRequirementOntology.owl#>
PREFIX rdfs: <http://www.w3.org/2000/01/rdf-schema#>
PREFIX owl:  <http://www.w3.org/2002/07/owl#>
PREFIX rdf:  <http://www.w3.org/1999/02/22-rdf-syntax-ns#>

SELECT DISTINCT ?xSegment
WHERE
{
    ?xSegment rdf:type base:Segment .
    ?xSegment base:Segment_length ?xSegment_length .

    FILTER (?xSegment_length <= 0)
}
\end{lstlisting}

\picScale{trainbenchmark-poslength}{The PosLength query's pattern}{0.4}

\subsubsection{Relational algebraic form} The \textit{PosLength} query can be formalized in relational algebra as:

$$ \pi_{\mathit{Sensor\_id}} \left( \sigma_{\mathit{Segment\_length} \leq 0} \left( \mathit{Sensor} \right) \right) $$

%%%%%%%%%%%%%%%%%%%%%%%%%%%%%%%%%%%%%%%%%%%%%%%%%%%%%%%%%%%%%%%%%%%%%%%%%%%%%%%
\subsection{RouteSensor}

The \textit{RouteSensor} query is discussed in~\autoref{routesensor}.

%%%%%%%%%%%%%%%%%%%%%%%%%%%%%%%%%%%%%%%%%%%%%%%%%%%%%%%%%%%%%%%%%%%%%%%%%%%%%%%
\subsection{SignalNeighbor}

\subsubsection{Description} The \textit{SignalNeighbor} well-formedness constraint requires that routes that are connected through sensors and track elements have to belong to the same signal. Therefore, the query (\figref{trainbenchmark-signalneighbor}) checks for routes which have an exit signal and a sensor connected to another sensor (which is in a definition of another route) by two track elements, but there is no other route that connects the same signal and the other sensor. The SPARQL representation of the query is shown in \lstref{signalneighbor-sparql}.

\subsubsection{Goal} This pattern checks for the absence of circles, so the efficiency of the join operation is tested. One-way navigable references are also present in the constraint, so the efficient evaluation of these are also tested. Subsumption inference is required, as the two track elements can be switches or segments.

\begin{lstlisting}[caption=The RouteSensor query in SPARQL, label=lst:signalneighbor-sparql]
PREFIX base: <http://www.semanticweb.org/ontologies/2011/1/TrainRequirementOntology.owl#>
PREFIX rdfs: <http://www.w3.org/2000/01/rdf-schema#>
PREFIX owl:  <http://www.w3.org/2002/07/owl#>
PREFIX rdf:  <http://www.w3.org/1999/02/22-rdf-syntax-ns#>

SELECT DISTINCT ?xRoute1
WHERE
{
  ?xRoute1 rdf:type base:Route .
  ?xSensor1 rdf:type base:Sensor .
  ?xSensor2 rdf:type base:Sensor .
  ?xSignal rdf:type base:Signal .
  ?xTrackElement1 rdf:type base:Trackelement .
  ?xTrackElement2 rdf:type base:Trackelement .
  
  ?xRoute1 base:Route_exit ?xSignal .
  ?xRoute1 base:Route_routeDefinition ?xSensor1 .
  ?xTrackElement1 base:TrackElement_sensor ?xSensor1 .
  ?xTrackElement1 base:TrackElement_connectsTo ?xTrackElement2 .
  ?xTrackElement2 base:TrackElement_sensor ?xSensor2 .
  
  ?xRoute3 rdf:type base:Route .
  ?xRoute3 base:Route_routeDefinition ?xSensor2 .
  FILTER ( ?xRoute3 != ?xRoute1 )
  
  OPTIONAL { 
           ?xRoute2 rdf:type base:Route .
           ?xRoute2 base:Route_entry ?xSignal .
           ?xRoute2 base:Route_routeDefinition ?xSensor2 .
           } .
  FILTER (!BOUND(?xRoute2))
}
\end{lstlisting}

\picScale{trainbenchmark-signalneighbor}{The SignalNeighbor query's pattern}{0.4}

\subsubsection{Relational algebraic form} The \textit{SignalNeighbor} query can be formalized in relational algebra as:

\begin{align*}
& \pi_{\mathit{Route\_entry.Route}} \big( \sigma_{\mathit{Route\_entry.Route} \neq \mathit{Route\_routeDefinition_2.Route}} \big( \\
& \quad \mathit{Route\_entry} \naturaljoin \mathit{Route\_routeDefinition_1} \naturaljoin \mathit{TrackElement\_sensor_1} \naturaljoin \\
& \quad \mathit{TrackElement\_connectsTo} \naturaljoin \mathit{TrackElement\_sensor_2} \naturaljoin \mathit{Route\_routeDefinition_2} \antijoin \\
& \quad \left( \mathit{Route\_exit} \naturaljoin \mathit{Route\_routeDefinition_3} \right) \\
& \big) \big)
\end{align*}

%%%%%%%%%%%%%%%%%%%%%%%%%%%%%%%%%%%%%%%%%%%%%%%%%%%%%%%%%%%%%%%%%%%%%%%%%%%%%%%
\subsection{SwitchSensor}

\subsubsection{Description} The \textit{SwitchSensor} well-formedness constraint requires that every switch must have at least one sensor connected to it. Therefore, the query (\figref{trainbenchmark-switchsensor}) checks for switches that have no sensors associated with them. The SPARQL representation of the query is shown in \lstref{switchsensor-sparql}.

\subsubsection{Goal} This query checks whether an object is connected to a relation. This pattern is common in more complex queries, e.g.\ it is used the \textit{RouteSensor} and the \textit{SignalNeighbor} queries.

\begin{lstlisting}[caption=The RouteSensor query in SPARQL, label=lst:switchsensor-sparql]
PREFIX base: <http://www.semanticweb.org/ontologies/2011/1/TrainRequirementOntology.owl#>
PREFIX rdfs: <http://www.w3.org/2000/01/rdf-schema#>
PREFIX owl:  <http://www.w3.org/2002/07/owl#>
PREFIX rdf:  <http://www.w3.org/1999/02/22-rdf-syntax-ns#>

SELECT DISTINCT ?xSwitch
WHERE
{
  ?xSwitch rdf:type base:Switch .

  OPTIONAL { 
           ?xSensor rdf:type base:Sensor .
           ?xSwitch base:TrackElement_sensor ?xSensor .
           } .
  FILTER (!BOUND(?xSensor))
}
\end{lstlisting}

\picScale{trainbenchmark-switchsensor}{The SwitchSensor query's pattern}{0.4}

\subsubsection{Relational algebraic form} The \textit{SwitchSensor} query can be formalized in relational algebra as:

$$ \mathit{Switch} \antijoin \mathit{Sensor} $$
