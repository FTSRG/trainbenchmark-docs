\chapter{Overview}

Scalability issues in model-driven engineering arise due to the increasing
complexity of modeling workloads. This complexity comes from two main factors:
(i) \emph{instance model sizes} are exhibiting a tremendous growth as the
complexity of systems-under-design is increasing, (ii) increasing \emph{feature
sophistication} in toolchains, such as complex model validation or
transformations.

One of the the most computationally expensive tasks in modeling applications are
\emph{model queries}. While there are a number of existing benchmarks for
queries over relational databases~\cite{tpc-c} or graph stores
\cite{BerlinBenchmark, SP2Bench}, modeling tool workloads are significantly
different. Specifically, modeling tools use much more complex queries than
typical transactional systems, and the real world performance is more affected by
response time (i.e.\ execution time for a specific operation such as validation
or transformation) than throughput (i.e.\ the amount of parallel transactions).


\section{Overview}
To address this challenge, the Train Benchmark~\cite{SCP2014,TBwebsite} is a macro
benchmark that aims to measure batch and incremental query evaluation
performance, in a scenario that is specifically modeled after \emph{model
validation} in (domain-specific) modeling tools: at first, the entire model is
validated, then after each model manipulation (e.g.\ the deletion of a
reference) is followed by an immediate re-validation. The benchmark measures
execution times for four phases:

\begin{enumerate}
 \item During the \emph{read} phase, the instance model
 is loaded from hard drive to memory. This includes the parsing of the input as
 well as initializing data structures of the tool. 
 
 \item In the \emph{check} phase, the instance model is queried to identify
 invalid elements. This can be as simple as reading the results from cache, or
 the model can be traversed based on some index. The result of this phase  
 is the set of the erroneous objects.
 
 \item In the \emph{edit} phase, the model is modified to simulate effects of
 manual user edits. Here the size of the change set can be adjusted to
 correspond to small manual edits as well as large model transformations.
 
 \item The re-validation of the model is carried out in the \emph{re-check}
 phase similarly to the \emph{check} phase.
\end{enumerate}

The Train Benchmark computes two derived results based on the recorded data:
(1) \emph{batch validation time} (the sum of the \emph{read} and \emph{check} phases)
represents the time that the user must wait to start to use the tool; (2)
\emph{incremental validation time} consists of the \emph{edit} and
\emph{re-check} phases performed 100 times, representing the time that the
user spent waiting for the tool validation.


\section{Instance Models}
The Train Benchmark uses a domain-specific model of a railway system that
originates from the \mbox{MOGENTES} EU FP7~\cite{Mogentes} project, where both the metamodel and the
well-formedness rules were defined by railway domain experts. This domain
enables the definition of both simple and more complex model queries while it is
simple enough to incorporate solutions from other technological spaces
(e.g.\ ontologies, relational databases and RDF). This allows the comparison of
the performance aspects of wider range of query tools from a constraint
validation viewpoint.

The instance models are systematically generated based on the metamodel and the
defined complex model queries: small instance model fragments are generated
based on the queries, then they are placed, randomized and connected to each
other. The methodology takes care of controlling the number of matches of all
defined model queries. To break symmetry, the exact number of elements and
cardinalities are randomized.
 
This brings artificially generated models \emph{closer to real world instances}
and \emph{prevents query tools from efficiently storing} or caching of instance
models. During the generation of the railway system model, errors are injected
at random positions. These errors can be found in the check phase of the
benchmark, which are reported and can be corrected during the edit phase. The
initial number of constraint violating elements are low (<1\% of total
elements).
 
\section{Queries and Transformations}
Queries are defined informally in plain text (in a tool independent way) and
also formalized using the standard OCL language as a reference implementation
(available on the benchmark website~\cite{TBwebsite}). The queries range from
simple attribute value checks to complex navigation operations consisting of
several (4+) joins.

The functionally equivalent variants of these queries are formalized using the
query language of different tools applying tool based optimizations. As a
result, all query implementations must return (the same set of) invalid instance
model elements.
 
In the \emph{edit} phase, the model is modified to change the result set to be
returned by the query in the \emph{recheck} phase. For simulating manual
modifications, the benchmark always performs 100 random edits (fixed low
constant) which increases the number of erroneous elements. An edit operation
only modifies a single model elements at once \todo{rewrite this} -- more complex model manipulation is
modelled as series of edits.

\section{Evaluation of Measurements}
The Train Benchmark defines a Java-based framework and application programming
interface that enables the integration of additional metamodels, instance
models, query implementations and even new benchmark scenarios (which may be
different from the original four-phase concept). The default implementation
contains a benchmark suite for queries implemented in Java, Eclipse OCL and
\eiq{}.

Measurements are recorded automatically in a machine-processable format (CSV)
that is automatically processed by R~\cite{TB:R} scripts. An extended version of the Train
Benchmark~\cite{TB:ASE2013} features several (instance model, query-specific and
combined) \emph{metrics} that can be used to characterize the ``difficulty'' of
benchmark cases numerically, and -- since they can be evaluated automatically
for other domain/model/query combinations -- allow to compare the benchmark
cases with other real-world workloads.
