\chapter{High-level Description}

\section{Benchmark Goal}

The main goal of the Train Benchmark is to measure the \emph{execution time} of graph-based query processing tools with particular emphasis on incremental query reevaluation. The execution time is measured against models of growing sizes generated by an instance model generator. This way, the scalability of the tools is evaluated.

The Train Benchmark also demonstrates other abilities of the tools, including transformation capabilities, conciseness of the query and transformation language, convenience of the API, compatibility with different metamodeling languages and so on.

\section{Methodology and Philosophy}

The Train Benchmark specifies a metamodel with a set of queries and transformation operations. The queries look for model elements that violate a particular well-formedness constraint. The queries are expected to return with the unique identifier of each invalid element.

